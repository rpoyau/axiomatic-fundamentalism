% main.tex
% Axiomatic Fundamentalism — General, framework-agnostic guideline
\documentclass[12pt]{article}

\usepackage[margin=1in]{geometry}
\usepackage{amsmath,amssymb}
\usepackage{hyperref}
\usepackage{enumitem}
\setlist{nosep}

\title{Axiomatic Fundamentalism:\\
A Concise Guide for Authors and Reviewers}
\author{}
\date{August 8, 2025}

\begin{document}
\maketitle

\begin{abstract}
This document sets a discipline for writing and reviewing work that proceeds \emph{only} from stated axioms. 
Every claim must be a logical consequence of those axioms; no external ``laws'' or ad hoc postulates are allowed as premises. 
All models are relational: distinctions generate relations; within a model nothing exists beyond declared distinctions, the relations they induce, and mathematics applied to them. 
Information-theoretic measures constrain representation and resolution; they are warnings to the modeller, not physical laws.
\end{abstract}

\section{Purpose and Scope}
\label{sec:purpose}
The goal is traceable, scope-tight derivations: state axioms once; derive consequences using mathematics; specialise only by explicit modelling choices; extract corollaries; propose falsification hooks. 
Avoid defending the axioms or digressing into edge cases beyond scope.

\section{Core Definitions}
\label{sec:definitions}
\begin{itemize}
  \item \textbf{Axiom} (\(\mathcal A\)): A precisely stated sentence taken as true within the document’s scope.
  \item \textbf{Consequence}: Any statement derived from \(\mathcal A\) by mathematics (algebra, calculus, logic, measure theory, differential geometry, \emph{etc.}) together with declared modelling choices.
  \item \textbf{Modelling choice}: A declared definition or constraint (variables, units, admissible sets, boundaries, norms/metrics, kernels, objective or cost functionals). A modelling choice is \emph{not} an empirical postulate.
  \item \textbf{External postulate/law}: A named principle imported from outside \(\mathcal A\) (e.g., “because Law X says so”). Disallowed as a premise.
\end{itemize}

\section{Core Commitments (Axiomatic Fundamentalism)}
\label{sec:commitments}
\begin{enumerate}[label=\textbf{AF\arabic*.}]
  \item \textbf{State axioms once, precisely.} Define symbols and scope. Do \emph{not} defend the axioms; simply state them.
  \item \textbf{Derive with mathematics only.} Immediate consequences follow from axioms via mathematics, not by appeal to external laws.
  \item \textbf{Specialise by declaration, not by postulate.} Introduce any constraint or structure as an explicit modelling choice.
  \item \textbf{No midstream imports.} Never justify a step by invoking an external “law” or tacit empirical rule.
  \item \textbf{No retrofitting.} Do not alter axioms or smuggle new premises to force a target result after derivations begin.
  \item \textbf{Scope discipline.} Exclude edge cases beyond scope instead of speculating; keep proofs local and self-contained.
  \item \textbf{Traceability \& falsifiability.} Each derived claim should indicate which axioms and choices it depends on and admit concrete tests that could contradict it.
\end{enumerate}

\section{Relational Foundations (Universal)}
\label{sec:relational}
All models are relational. Distinctions generate relations; within a model nothing exists beyond declared distinctions, the relations they induce, and mathematics applied to them.

\paragraph{Glossary.}
\begin{itemize}
  \item \textbf{Target}: The system being modelled.
  \item \textbf{Distinction}: A declared differentiating property or partition on outcomes.
  \item \textbf{Relation}: A mapping/predicate constructed from distinctions (e.g., adjacency, order, metric).
  \item \textbf{Model universe}: Everything admitted to exist \emph{within} the model—declared distinctions, induced relations, and mathematical constructions thereof.
\end{itemize}

\paragraph{Relational foundations (RF).}
\begin{enumerate}[label=\textbf{RF\arabic*.}]
  \item \textbf{Distinction precedes relation.} All relations in the model are generated from declared distinctions.
  \item \textbf{Closure.} Within the model universe, nothing else exists: no distinction \(\Rightarrow\) no relation; no relation \(\Rightarrow\) no model entity.
  \item \textbf{No absolutes.} Undeclared “absolute properties” cannot enter derivations.
  \item \textbf{Information is bookkeeping, not a law.} Information-theoretic measures constrain what can be represented or resolved; they are warnings to the modeller, not physical postulates.
  \item \textbf{No elevation of limitations.} Do not promote modelling limitations (resolution, coding, partition choice) to universal laws.
\end{enumerate}

\section{Protocol for Authors}
\label{sec:protocol}
\begin{enumerate}
  \item \textbf{State the axiom(s).} One place; final form; symbols and domains fixed.
  \item \textbf{Fix scope.} What the axiom covers and what is explicitly out of scope. No defence essays.
  \item \textbf{Derive immediate consequences.} From axioms by mathematics only. Keep proofs local.
  \item \textbf{Declare modelling choices.} Explicitly define any resources, domains, geometries, metrics, mobilities, gauges/units, cost functionals.
  \item \textbf{Extract corollaries.} From consequences + declared choices, by algebra/calculus/variation. If a result resembles a known “law,” present it as \emph{derived}; never cite the law as a premise.
  \item \textbf{Demonstrations.} Provide examples/toy models that instantiate earlier steps; introduce no new premises.
  \item \textbf{Falsification hooks.} List concrete tests that would contradict specific derived statements.
\end{enumerate}

\section{Allowed vs.\ Disallowed Moves}
\label{sec:moves}
\paragraph{Allowed (as premises).}
\begin{itemize}
  \item Pure mathematics and logic.
  \item Definitions of symbols, units, admissible sets, boundaries, coordinate choices, norms, metrics.
  \item Declared parameterisations and cost functionals (kernels, mobilities, priors, topologies), introduced as modelling choices.
\end{itemize}

\paragraph{Disallowed (as premises).}
\begin{itemize}
  \item Importing external “laws” mid-derivation.
  \item Retrofitting the axiom or adding hidden assumptions to force agreement with a target result.
  \item Defensive essays about plausibility of the axiom.
  \item Edge-case digressions beyond scope; mark them out-of-scope instead.
\end{itemize}

\section{Structure and Labelling (Mechanics)}
\label{sec:mechanics}
\begin{itemize}
  \item Use clear hierarchy (Sections, Subsections). Give important claims and equations unique labels.
  \item References must point to a single, precise anchor—avoid vague ranges like “see §§??–??”.
  \item Keep notation and units consistent; (re)declare gauge/units once per section if needed.
  \item Mark modelling choices explicitly as \emph{Definition} or \emph{Choice} at first use.
\end{itemize}

\section{Reviewer Rubric (Binary Checks)}
\label{sec:rubric}
\begin{enumerate}
  \item Are axioms stated once, precisely, with symbols defined and scope fixed?
  \item Do immediate consequences follow from the axiom(s) and mathematics only?
  \item Are modelling choices declared explicitly (not smuggled as postulates)?
  \item Are corollaries proved without importing external laws?
  \item Do demonstrations instantiate earlier steps only?
  \item Are falsification items tied to specific derived statements?
  \item Are labels unique and references precise; is notation/units consistent?
  \item Are relational foundations RF1–RF5 respected?
\end{enumerate}

\section{Minimal Author Templates}
\label{sec:templates}

\paragraph{Axiom (template).}
\begin{quote}
\textbf{Axiom A1 (Name).} \emph{Formal statement of \(\mathcal A\).}\hfill[\textit{Scope:} …]
\end{quote}

\paragraph{Immediate consequence (template).}
\begin{quote}
\textbf{Consequence C1.} \emph{Statement.}\\
\textit{Proof sketch.} From A1 by [mathematical steps] only.
\end{quote}

\paragraph{Modelling choice (template).}
\begin{quote}
\textbf{Choice D1.} \emph{We define} [objects/sets/constraints] \emph{to be} … \hfill[\textit{Rationale:} clarity only]
\end{quote}

\paragraph{Corollary (template).}
\begin{quote}
\textbf{Corollary K1.} \emph{Statement.}\\
\textit{Derivation.} From C1 and D1 by [algebra/calculus/variation]; no external postulates.
\end{quote}

\section{Release and Versioning}
\label{sec:release}
Maintain a short changelog noting added axioms, altered scopes, and new modelling choices. Any change to an axiom increments the document’s major version.

\end{document}
